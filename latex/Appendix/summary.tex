\chapter*{Tóm tắt}
\label{summary}
% Trình bày tóm tắt vấn đề nghiên cứu, các hướng tiếp cận, cách giải quyết vấn đề và một số kết quả đạt được.

\noindent Con người luôn mong muốn giảm bớt gánh nặng trong việc suy nghĩ và lựa chọn, doanh nghiệp thì luôn muốn bán được nhiều sản phẩm hơn. Xây dựng hệ thống gợi ý là bài toán không mới, tuy nhiên việc xây dựng sao cho hiệu quả thực sự vẫn đang là một bài toán tồn tại rất nhiều vấn đề vẫn chưa thể giải quyết triệt để. Rất nhiều các phương pháp được đề xuất, nhìn chung ta vẫn đang phải đối mặt với các vấn đề như dữ liệu thưa hay việc học biểu diễn từ đồ thị tương tác người dùng -- sản phẩm đang chịu ảnh hưởng nhiều bởi nhiễu. Điều này thúc đẩy động lực cho các phương pháp học mới ra đời với mục tiêu là hạn chế tác động của dữ liệu thưa và tăng độ chịu nhiễu của mô hình.

Mạng học sâu đồ thị đã chứng minh được tính hiệu quả khi áp dụng lên hệ thống gợi ý dựa trên việc khai thác cấu trúc đồ thị tương tác của người dùng. Hầu hết các bộ dữ liệu được sử dụng để huấn luyện các mô hình học gợi ý đều rất thưa, điều này dẫn tới việc huấn luyện và đánh giá các mô hình này đều cho kết quả không tốt. Mô hình Học tự giám sát ra đời và được áp dụng rất thành công trong lĩnh vực thị giác máy tính và xử lý ngôn ngữ tự nhiên và có tiềm năng rất lớn trong cả lĩnh vực học đồ thị.

Khóa luận này nhằm mục đích nghiên cứu các vấn đề xoay quanh việc áp dụng \textbf{Học tự giám sát lên mạng học sâu đồ thị cho hệ thống gợi ý}. Bằng cách sử dụng Học tương phản -- một hướng tiếp cận của Học tự giám sát bổ trợ cho tác vụ gợi ý, thuật toán đề xuất của khóa luận đã đạt hiệu quả cao trên những tập dữ liệu rất thưa khi so sánh với các mô hình sử dụng mạng học sâu đồ thị cho hệ thống gợi ý khác.
